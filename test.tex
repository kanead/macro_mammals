\documentclass{article}

\begin{document}


\title{Differential response of volant and non-volant small mammal communities to spatial gradient}
\begin{titlepage}
\maketitle
\end{titlepage}






\section{Introduction}
Patterns of biodiversity, such as species richness, vary across space and understanding this has remained a major focus of ecology (Ricklefs, 1987; Yang et al., 2018). For example, research along latitudinal or elevational gradients is still topical and popular (Batista et al., 2020; Etienne et al., 2019; Fieldsend, 2019; Rahbek et al., 2019). Yet, the processes acting on patterns of diversity, particularly at large spatial scales such as across continents, remain poorly understood (Herkt et al., 2016; Keil \& Chase, 2019). 

\vspace{5mm}


Spatial patterns in species richness can best be understood by decomposing diversity into three inter-related components: alpha, beta, and gamma (Bennett \& Gilbert, 2016; Gardener, 2014; Whittaker, 1960). Alpha diversity refers to the number of species at a single locality, while gamma diversity is the species richness across a region. Beta diversity captures the change in species composition between sites, where the product of alpha and beta diversity gives gamma diversity (Andrés Baselga, 2010). As a result, the study of beta diversity is relevant for understanding patterns of species richness across macroecological scales (Anderson et al., 2011; Lennon et al., 2001; Mcknight et al., 2007; Varzinczak et al., 2018). 


\vspace{5mm}
Three main hypotheses have been put forward to explain patterns of beta diversity across broad continent scales (Varzinczak et al., 2018). First, environmental or niche-based processes may determine the ecological conditions suitable for a species to persist at a particular site (Mathew A. Leibold \& Mikkelson, 2002). Second, spatial processes may differentially affect the ability of species to disperse and hence to occupy geographically distant regions (M. A. Leibold et al., 2004). Finally, historical or biogeographical processes such as glaciation events may have influenced species distributions (Couvreur et al., 2020). These three hypotheses are not necessarily mutually exclusive but identifying their relative contributions to species distribution patterns may provide useful evolutionary and macroecological insights (Batista et al., 2020).

\vspace{5mm}

Beta diversity, however, can be further partitioned into turnover, which is a measure of the replacement of species along a gradient, and nestedness, which reflects the loss of species leading to species poor sites being a nested subset of species rich sites (Andrés Baselga, 2010). In this way, the study of beta diversity has provided the basis for testing hypotheses related to the determinants of species distributions and hence offers novel insights into the patterns of biodiversity (A. Baselga \& Orme, 2012; Moura et al., 2017; Varzinczak et al., 2018). Turnover and nestedness may vary depending on environmental gradients. For example, turnover declines while nestedness increases with latitude, presumably due to harsher environmental conditions near the poles (Qian \& Xiao, 2012; Soininen et al., 2018). In contrast, there is a strong longitudinal effect in species turnover of South American rodents, as a result of the elevational gradient created by Andes (Maestri \& Patterson, 2016). Turnover and nestedness may also be affected by species traits;  species that are good dispersers are expected to have lower turnover due to their ability to move across the landscape (Qian \& Ricklefs, 2012; Soininen et al., 2018). 

\vspace{5mm}

The diversity of biological communities has most commonly been studied in terms of species or taxonomic richness, which deals with species entities (Maestri \& Patterson, 2016; Melo et al., 2009). For example, taxonomic species richness is simply a count of the number of species at a site. However, the diversity of a community can also be measured in terms of phylogenetic and functional diversity (García-Navas, 2019; Mcknight et al., 2007; Penone et al., 2016). Phylogenetic richness takes into account the evolutionary (systematic) relationships between species and is commonly calculated as the total branch lengths of a phylogenetic tree that connect the group of species in question (Chao et al., 2014; Varzinczak et al., 2018). In contrast, functional richness can be viewed in terms of how species in a community vary in terms of traits that may include morphological, behavioural, or physiological features. 

\vspace{5mm}

By combining all three measures of beta diversity (taxonomic, phylogenetic, and functional), it is possible to test hypotheses related to the processes that have created the broad-scale patterns of diversity at macroecological scales (Safi et al., 2011). These three measures are closely related and therefore are expected to show similar broad trends; however, it is where their trends deviate or become “decoupled” that they become most informative (Penone et al., 2016). In particular, functional beta diversity is expected to be low across sites with similar environments due to convergent evolution selecting similar morphological adaptations at these sites (García-Navas, 2019). In contrast, phylogenetic diversity is expected to be highest across sites that are linked within a continuous region without severe geographical barriers (Penone et al., 2016). Hence, regions with similar environments but separated by biogeographical barriers are expected to have low functional beta diversity but high phylogenetic beta diversity. Conversely, contiguous regions without barriers but under vastly different environmental conditions are expected to have high functional beta diversity and low phylogenetic diversity. 

\vspace{5mm}

Sub-Saharan Africa (Afrotropical region) has a diverse mammalian community that includes over 1570 species (ca. 24\% of global mammal richness), marginally second only to the Neotropics (Burgin et al., 2018). The three mammalian orders with the greatest number of species in Africa are the rodents (Rodentia), bats (Chiroptera), and shrews and allies (Eulipotyphla) (Kingdon et al., 2013). These small mammals are also abundant, occurring at high densities and play critical roles in ecosystem functioning (Fischer et al., 2018; Kunz et al., 2011; Lacher et al., 2016; Russo \& Jones, 2015; Tschumi et al., 2018). 

\vspace{5mm}
Of the three small mammal groups, African species richness has only been mapped for bats (Herkt et al., 2016; A. Monadjem et al., 2018), which shows peaks in the equatorial zone, particularly at the transition between the rainforest and savanna biomes (Fahr \& Kalko, 2011), and tropical uplands (Ara Monadjem et al., 2016). The distribution of African species richness is not known for rodents and shrews, but is thought to peak in the rainforest zone, particularly in half a dozen or so so-called refugia where mammals were thought to have survived forest contraction during the last glaciation event (Happold, 1996). In contrast to patterns of species richness, studies of the patterns of beta diversity are almost entirely lacking for African small mammals. On a global scale, phylogenetic beta diversity for rodents and bats appears to be highest in the north of the continent, but south of the Sahara desert, with an uneven distribution across the rest of the continent (Peixoto et al., 2017). On an African continental scale, phylogenetic beta diversity of muroid rodents between bioregions is high with turnover more important than nestedness; in contrast, the relative contributions of turnover and nestedness were similar for functional beta diversity (García-Navas, 2019). 

\vspace{5mm}
In this study we compare turnover and nestedness in three groups of small mammals across sub-Saharan Africa: bats (volant), rodents (non-volant) and insectivores (non-volant). Specifically, we ask whether dispersal ability (i.e. the capability to fly) affect beta diversity metrics. We expected that bats, due to their ability to fly, would be better dispersers across the landscape, which would translate into lower turnover rates of beta diversity (Varzinczak et al., 2019). In contrast, we predicted that rodents and shrews would be inferior dispersers compared to bats, and therefore demonstrate higher rates of turnover. We made a further prediction that shrews, due to their generally smaller body sizes and high metabolic rates (Gliwicz \& Taylor, 2002) would have higher turnover rates than rodents. Finally, we ask whether the patterns of taxonomic, phylogenetic, and functional beta diversity differ across the region and whether these can be linked to specific bioregions.  






\end{document}